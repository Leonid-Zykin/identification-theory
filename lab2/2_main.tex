\chapter{Динамические методы идентификации}

\section{Цель работы и исходные данные}
Целью работы является изучение динамических методов идентификации параметров систем, включая градиентные алгоритмы с постоянным коэффициентом усиления и рекуррентный метод наименьших квадратов.

\section{Краткие теоретические сведения}
Рассматривается модель линейной регрессии
\begin{equation}
    y(k) = \Phi^{\mathsf T}(k)\theta^*,
\end{equation}
где $y(k) \in \mathbb{R}^1$~--- выходной сигнал, $\Phi(k) \in \mathbb{R}^m$~--- вектор регрессоров, $\theta^* \in \mathbb{R}^m$~--- вектор неизвестных параметров.

В отличие от статических методов, динамические методы идентификации обновляют оценки параметров по мере поступления новых измерений. Для дискретного времени рекуррентный алгоритм имеет вид:
\begin{equation}
    \hat{\theta}(k) = A\{\hat{\theta}(k-1), \Phi(k), y(k)\},
\end{equation}
где $A$~--- алгоритм обновления оценки параметров.

\subsection{Градиентные методы с постоянным коэффициентом усиления}
Квадратичный критерий качества:
\begin{equation}
    J_{\text{SE}}(k) = \frac{1}{2}(y(k) - \Phi^{\mathsf T}(k)\hat{\theta}(k))^2.
\end{equation}
Градиентный алгоритм идентификации в дискретном времени:
\begin{equation}
    \hat{\theta}(k) = \hat{\theta}(k-1) + \gamma\frac{\Phi(k)e^0(k)}{1 + \gamma\Phi^{\mathsf T}(k)\Phi(k)},
\end{equation}
где $e^0(k) = y(k) - \Phi^{\mathsf T}(k)\hat{\theta}(k-1)$, $\gamma > 0$~--- коэффициент усиления.

Упрощенный градиентный алгоритм:
\begin{equation}
    \hat{\theta}(k) = \hat{\theta}(k-1) + \gamma\Phi(k)e^0(k),
\end{equation}
который обеспечивает устойчивость только для достаточно малых значений $\gamma$.

\subsection{Идентификация динамических систем в дискретном времени}
Модель ARX (AutoRegressive eXogenous):
\begin{equation}
    A(q^{-1})y(k) = B(q^{-1})u(k) + \eta(k),
\end{equation}
где $A(q^{-1}) = 1 + a_1q^{-1} + \ldots + a_nq^{-n}$, $B(q^{-1}) = b_0 + b_1q^{-1} + \ldots + b_mq^{-m}$, $q^{-1}$~--- оператор сдвига назад ($q^{-1}y(k) = y(k-1)$).

В форме линейной регрессии:
\begin{equation}
    y(k) = \Phi^{\mathsf T}(k)\theta^* + \eta(k),
\end{equation}
где $\Phi(k) = [-y(k-1), \ldots, -y(k-n), u(k), \ldots, u(k-m)]^{\mathsf T}$, $\theta^* = [a_1, \ldots, a_n, b_0, \ldots, b_m]^{\mathsf T}$.

\subsection{Идентификация динамических систем в непрерывном времени}
Для непрерывной системы вида $y(t) = \frac{b_0}{p+a_0}u(t) + \eta(t)$, где $p$~--- оператор дифференцирования, при пренебрежимо малой помехе $\eta(t) = 0$ и доступности производной $\dot{y}(t)$ модель может быть представлена в виде линейной регрессии:
\begin{equation}
    \dot{y}(t) = [-y(t)\quad u(t)]\begin{bmatrix} a_0 \\ b_0 \end{bmatrix}.
\end{equation}

\section{Задание 1}
\subsection{Постановка}
Требуется идентифицировать параметры дискретной линейной системы с передаточной функцией $W(z) = \frac{b}{z+a}$ методом градиентного алгоритма. Параметры системы: $a = 0.96$, $b = 2.5$, частота входного сигнала $\omega = 3.14$~рад/с. Интервал дискретизации $T_d = 0.1$~с. Время симуляции составляет $30$~с для обеспечения достаточной сходимости оценок параметров.

\subsection{Методика}
Дискретная модель системы в форме линейной регрессии:
\begin{equation}
    y(k) = -a y(k-1) + b u(k-1) + \eta(k),
\end{equation}
где вектор регрессоров $\Phi(k) = [-y(k-1), u(k-1)]^{\mathsf T}$, вектор параметров $\theta^* = [a, b]^{\mathsf T}$.

Градиентный алгоритм (3):
\begin{equation}
    \hat{\theta}(k) = \hat{\theta}(k-1) + \gamma\frac{\Phi(k)e^0(k)}{1 + \gamma\Phi^{\mathsf T}(k)\Phi(k)},
\end{equation}
где $e^0(k) = y(k) - \Phi^{\mathsf T}(k)\hat{\theta}(k-1)$.

Упрощенный градиентный алгоритм (4):
\begin{equation}
    \hat{\theta}(k) = \hat{\theta}(k-1) + \gamma\Phi(k)e^0(k).
\end{equation}

\subsection{Результаты}


\begin{figure}[H]
    \centering
    \includegraphics{images/task1/gradient3_gamma1.png}
    \caption{Задание~1: градиентный алгоритм (3) с $\gamma = 1$}
    \label{fig:zad1_grad3_g1}
\end{figure}

\begin{figure}[H]
    \centering
    \includegraphics{images/task1/gradient3_gamma3.png}
    \caption{Задание~1: градиентный алгоритм (3) с $\gamma = 3$}
    \label{fig:zad1_grad3_g3}
\end{figure}

\begin{figure}[H]
    \centering
    \includegraphics{images/task1/gradient3_gamma10.png}
    \caption{Задание~1: градиентный алгоритм (3) с $\gamma = 10$}
    \label{fig:zad1_grad3_g10}
\end{figure}

\begin{figure}[H]
    \centering
    \includegraphics{images/task1/simple4_gamma0.5.png}
    \caption{Задание~1: упрощенный алгоритм (4) с $\gamma = 0.5$}
    \label{fig:zad1_simple4_g05}
\end{figure}

\begin{figure}[H]
    \centering
    \includegraphics{images/task1/simple4_gamma10.png}
    \caption{Задание~1: упрощенный алгоритм (4) с $\gamma = 10$}
    \label{fig:zad1_simple4_g10}
\end{figure}

\subsection{Выводы}
Анализ результатов показывает влияние коэффициента усиления $\gamma$ на процесс идентификации. 

Для градиентного алгоритма (3) увеличение $\gamma$ ускоряет сходимость оценок параметров к истинным значениям. При $\gamma = 1$ получены оценки $\hat{a} = 0.818$, $\hat{b} = 2.292$ (истинные: $a^* = 0.96$, $b^* = 2.5$), что показывает неполную сходимость за время симуляции. При $\gamma = 3$ оценки улучшаются до $\hat{a} = 0.941$, $\hat{b} = 2.471$, а при $\gamma = 10$ практически достигают истинных значений: $\hat{a} = 0.960$, $\hat{b} = 2.500$. Алгоритм остаётся стабильным при всех значениях $\gamma$ благодаря нормализации в знаменателе формулы алгоритма.

Упрощенный алгоритм (4) демонстрирует принципиально иное поведение: при малых значениях $\gamma = 0.5$ алгоритм работает стабильно, но сходится значительно медленнее, чем алгоритм (3). За время симуляции $30$~с получены оценки $\hat{a} = 0.731$, $\hat{b} = 2.159$, которые ещё не достигли истинных значений, но показывают устойчивую тенденцию к сходимости (см. рисунок~\ref{fig:zad1_simple4_g05}). При увеличении времени симуляции до $120$~с оценки приближаются к истинным значениям: $\hat{a} \approx 0.952$, $\hat{b} \approx 2.488$, что подтверждает сходимость алгоритма, хотя и с меньшей скоростью по сравнению с алгоритмом (3). Однако при $\gamma = 10$ система становится неустойчивой уже через несколько шагов (примерно $0.5$~с), и оценки параметров экспоненциально расходятся до бесконечно больших значений. Это подтверждает теоретическое утверждение о том, что упрощенный алгоритм (4) обеспечивает устойчивость только для достаточно малых значений параметра $\gamma$, в отличие от алгоритма (3), который остаётся стабильным при любых $\gamma > 0$.

\section{Задание 2}
\subsection{Постановка}
Требуется идентифицировать параметры дискретной линейной системы с передаточной функцией $W(z) = \frac{b}{z^2 + a_1z + a_2}$ методом градиентного алгоритма. Параметры системы: $a_1 = -1.92$, $a_2 = 0.9215$, $b = 2.1$, частота входного сигнала $\omega = 46.49$~рад/с. Интервал дискретизации $T_d = 0.1$~с.

\subsection{Методика}
Дискретная модель системы в форме линейной регрессии:
\begin{equation}
    y(k) = -a_1 y(k-1) - a_2 y(k-2) + b u(k-1) + \eta(k),
\end{equation}
где вектор регрессоров $\Phi(k) = [-y(k-1), -y(k-2), u(k-1)]^{\mathsf T}$, вектор параметров $\theta^* = [a_1, a_2, b]^{\mathsf T}$.

Используется градиентный алгоритм (3) с $\gamma = 1$ для двух различных входных сигналов:
\begin{align}
    u_1(t) &= \sin(\omega t), \\
    u_2(t) &= \sin(\omega t) + 0.2\sin(0.5\omega t).
\end{align}

\subsection{Результаты}
Результаты идентификации представлены на рисунках~\ref{fig:zad2_sin} и~\ref{fig:zad2_mixed}.

\begin{figure}[H]
    \centering
    \includegraphics{images/task2/sin_input.png}
    \caption{Задание~2: идентификация с входным сигналом $u(t) = \sin(\omega t)$}
    \label{fig:zad2_sin}
\end{figure}

\begin{figure}[H]
    \centering
    \includegraphics{images/task2/mixed_input.png}
    \caption{Задание~2: идентификация с входным сигналом $u(t) = \sin(\omega t) + 0.2\sin(0.5\omega t)$}
    \label{fig:zad2_mixed}
\end{figure}

\subsection{Выводы}
Сравнение результатов идентификации показывает влияние состава входного сигнала на качество оценивания параметров. 

При входном сигнале $u(t) = \sin(\omega t)$ получены оценки: $\hat{a}_1 = -1.915$, $\hat{a}_2 = 0.922$, $\hat{b} = 2.095$ (истинные значения: $a_1^* = -1.92$, $a_2^* = 0.9215$, $b^* = 2.1$). Ошибки оценивания не превышают $0.005$ для всех параметров.

При смешанном входном сигнале $u(t) = \sin(\omega t) + 0.2\sin(0.5\omega t)$ получены оценки: $\hat{a}_1 = -1.916$, $\hat{a}_2 = 0.922$, $\hat{b} = 2.096$, что показывает незначительное улучшение точности идентификации. Смешанный входной сигнал обеспечивает более богатое возбуждение системы за счёт наличия двух частотных составляющих, что способствует более точной идентификации параметров.

\section{Задание 3}
\subsection{Постановка}
Требуется идентифицировать параметры непрерывной линейной системы $y(t) = \frac{b}{p+a}u(t)$ методом градиентного алгоритма в непрерывном времени. Параметры системы: $a = 0.9$, $b = 3.2$, частота входного сигнала $\omega = 7.53$~рад/с. Время симуляции составляет $30$~с, шаг интегрирования $dt = 0.01$~с.

\subsection{Методика}
При пренебрежимо малой помехе $\eta(t) = 0$ и доступности производной $\dot{y}(t)$ модель представляется в виде линейной регрессии:
\begin{equation}
    \dot{y}(t) = [-y(t)\quad u(t)]\begin{bmatrix} a \\ b \end{bmatrix}.
\end{equation}

Градиентный алгоритм в непрерывном времени (2):
\begin{equation}
    \frac{d}{dt}\hat{\theta}(t) = \gamma\Phi(t)e(t),
\end{equation}
где $e(t) = y(t) - \Phi^{\mathsf T}(t)\hat{\theta}(t)$, $\Phi(t) = [-y(t), u(t)]^{\mathsf T}$.

\subsection{Результаты}
Результаты идентификации представлены на рисунках~\ref{fig:zad3_g1}--\ref{fig:zad3_g10}.

\begin{figure}[H]
    \centering
    \includegraphics{images/task3/gamma1.png}
    \caption{Задание~3: градиентный алгоритм (2) с $\gamma = 1$}
    \label{fig:zad3_g1}
\end{figure}

\begin{figure}[H]
    \centering
    \includegraphics{images/task3/gamma3.png}
    \caption{Задание~3: градиентный алгоритм (2) с $\gamma = 3$}
    \label{fig:zad3_g3}
\end{figure}

\begin{figure}[H]
    \centering
    \includegraphics{images/task3/gamma10.png}
    \caption{Задание~3: градиентный алгоритм (2) с $\gamma = 10$}
    \label{fig:zad3_g10}
\end{figure}

\subsection{Выводы}
Анализ результатов показывает влияние коэффициента усиления $\gamma$ на скорость сходимости оценок параметров в непрерывном времени. 

При $\gamma = 1$ получены оценки: $\hat{a} = 0.860$, $\hat{b} = 3.194$ (истинные значения: $a^* = 0.9$, $b^* = 3.2$). Ошибки оценивания составляют $0.040$ и $0.006$ соответственно.

При $\gamma = 3$ оценки улучшаются: $\hat{a} = 0.899$, $\hat{b} = 3.197$, ошибки снижаются до $0.001$ и $0.003$.

При $\gamma = 10$ оценки практически совпадают с истинными: $\hat{a} = 0.899$, $\hat{b} = 3.197$, что подтверждает эффективность увеличения коэффициента усиления для ускорения сходимости. Увеличение $\gamma$ ускоряет процесс идентификации без потери устойчивости, в отличие от упрощенного алгоритма (4) в дискретном времени.

\section{Заключение}
В ходе работы изучены динамические методы идентификации параметров систем. Показано влияние коэффициента усиления $\gamma$ на процесс идентификации, а также важность выбора входного сигнала для обеспечения достаточного возбуждения системы. Градиентные алгоритмы с постоянным коэффициентом усиления обеспечивают простую реализацию, но требуют тщательного выбора параметра $\gamma$ для обеспечения устойчивости и быстрой сходимости.

