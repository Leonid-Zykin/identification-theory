% !TEX TS-program = pdflatex
\documentclass[a4paper,12pt]{article}

\usepackage[utf8]{inputenc}
\usepackage[T2A]{fontenc}
\usepackage[russian]{babel}
\usepackage{amsmath,amssymb,amsfonts}
\usepackage{geometry}
\geometry{margin=2cm}

\begin{document}

\section*{Условие неисчезающего возбуждения: матричное неравенство}

Рассматривается регрессионная модель
\[
    y(t) = \varphi^\top(t)\,\theta^\star + v(t),
\]
где $\varphi(t)\in\mathbb{R}^n$~--- вектор регрессора, $\theta^\star\in\mathbb{R}^n$~--- истинный вектор параметров, $v(t)$~--- помеха. В алгоритмах адаптивной идентификации ключевую роль играет условие \emph{неисчезающего возбуждения} (persistent excitation)
\begin{equation}
    \int_{t}^{t+T} \varphi(\tau)\varphi^\top(\tau)\,d\tau \;\geq\; \alpha I,
    \qquad \forall t,\quad T>0,\ \alpha>0,
    \label{eq:PE}
\end{equation}
где $I$~--- единичная матрица размера $n\times n$.

\subsection*{Почему левая часть~\eqref{eq:PE}~--- это матрица}

Вектор регрессора имеет вид
\[
    \varphi(t) =
    \begin{bmatrix}
        \varphi_1(t)\\
        \vdots      \\
        \varphi_n(t)
    \end{bmatrix}
    \in\mathbb{R}^n.
\]
Тогда произведение $\varphi(\tau)\varphi^\top(\tau)$~--- это \emph{внешнее произведение} вектора на себя:
\[
    \varphi(\tau)\varphi^\top(\tau)
    =
    \begin{bmatrix}
        \varphi_1(\tau)\\
        \vdots\\
        \varphi_n(\tau)
    \end{bmatrix}
    \begin{bmatrix}
        \varphi_1(\tau) & \dots & \varphi_n(\tau)
    \end{bmatrix}
    =
    \bigl[\varphi_i(\tau)\varphi_j(\tau)\bigr]_{i,j=1}^n.
\]
Каждый элемент этой матрицы есть скаляр $\varphi_i(\tau)\varphi_j(\tau)$, поэтому
\[
    \varphi(\tau)\varphi^\top(\tau)\in\mathbb{R}^{n\times n}.
\]
Интегрирование по времени в~\eqref{eq:PE} выполняется \emph{покомпонентно}:
\[
    \left(\int_{t}^{t+T} \varphi(\tau)\varphi^\top(\tau)\,d\tau\right)_{ij}
    =
    \int_{t}^{t+T} \varphi_i(\tau)\varphi_j(\tau)\,d\tau.
\]
То есть мы интегрируем каждый элемент матрицы по отдельности и снова получаем матрицу размера $n\times n$. Таким образом,
\[
    \int_{t}^{t+T} \varphi(\tau)\varphi^\top(\tau)\,d\tau \in \mathbb{R}^{n\times n},
\]
поэтому сравнение в~\eqref{eq:PE} является именно \emph{матричным} неравенством.

Отметим также, что для любого фиксированного $\tau$ матрица
$\varphi(\tau)\varphi^\top(\tau)$ является симметричной и неотрицательно
определённой (positive semi-definite), так как
\[
    x^\top\varphi(\tau)\varphi^\top(\tau)x
    = \bigl(\varphi^\top(\tau)x\bigr)^2 \ge 0
    \qquad \forall x\in\mathbb{R}^n.
\]
Интеграл суммы таких матриц также остаётся симметричной неотрицательно определённой матрицей.

\subsection*{Смысл неравенства матриц в~\eqref{eq:PE}}

Запись
\[
    A \;\geq\; B
\]
для симметричных матриц $A,B\in\mathbb{R}^{n\times n}$ понимается в смысле \emph{положительной полуопределённости} разности:
\[
    A \ge B
    \quad\Longleftrightarrow\quad
    A-B \text{ неотрицательно определена}
    \quad\Longleftrightarrow\quad
    x^\top(A-B)x \ge 0,\ \forall x\in\mathbb{R}^n.
\]
Если же дополнительно выполняется
$x^\top(A-B)x>0$ для всех ненулевых $x\neq0$, то матрица $A-B$
\emph{положительно определена} и пишут $A>B$.

Применяя это к условию~\eqref{eq:PE}, получаем
\[
    \int_{t}^{t+T} \varphi(\tau)\varphi^\top(\tau)\,d\tau \;\geq\; \alpha I
    \quad\Longleftrightarrow\quad
    \int_{t}^{t+T} \varphi(\tau)\varphi^\top(\tau)\,d\tau - \alpha I
    \text{ неотрицательно определена}.
\]
Эквивалентная запись через квадратичную форму:
\begin{equation}
    x^\top\!\left(
        \int_{t}^{t+T} \varphi(\tau)\varphi^\top(\tau)\,d\tau
        - \alpha I
    \right)\!x
    \;\geq\; 0
    \qquad \forall x\in\mathbb{R}^n,\ \forall t.
    \label{eq:PE-quadratic}
\end{equation}
Вынесем $x$ из интеграла:
\[
    x^\top\!\left(
        \int_{t}^{t+T} \varphi(\tau)\varphi^\top(\tau)\,d\tau
    \right)\!x
    =
    \int_{t}^{t+T} x^\top\varphi(\tau)\varphi^\top(\tau)x\,d\tau
    =
    \int_{t}^{t+T} \bigl(\varphi^\top(\tau)x\bigr)^2 d\tau.
\]
Тогда неравенство~\eqref{eq:PE-quadratic} можно переписать как
\[
    \int_{t}^{t+T} \bigl(\varphi^\top(\tau)x\bigr)^2 d\tau
    \;\geq\; \alpha \,\|x\|^2
    \qquad \forall x,\ \forall t.
\]
То есть для любого ненулевого направления $x$ энергия скалярного сигнала
$\varphi^\top(\tau)x$ на любом окне длины $T$ ограничена снизу положительной постоянной, пропорциональной $\|x\|^2$. Именно это интерпретируется как \emph{неисчезающее возбуждение}: регрессор $\varphi(t)$ не ``забывает'' ни одно направление в пространстве параметров.

\subsection*{Связь с положительной определённостью}

Из линейной алгебры известно, что для симметричной матрицы $M$ следующие условия эквивалентны:
\begin{enumerate}
    \item $M$ положительно определена;
    \item все собственные значения $M$ положительны;
    \item $x^\top M x > 0$ для всех $x\neq 0$.
\end{enumerate}
Если переписать~\eqref{eq:PE} как
\[
    \int_{t}^{t+T} \varphi(\tau)\varphi^\top(\tau)\,d\tau - \alpha I \;\ge 0,
\]
то требование ``$\ge$'' означает именно неотрицательную (а в усиленной форме --- положительную) определённость этой разности. Говоря словами преподавателя: \emph{``матрица слева $\ge$ матрицы справа'' эквивалентно утверждению, что матрица
    $\displaystyle
    \int_{t}^{t+T} \varphi(\tau)\varphi^\top(\tau)\,d\tau - \alpha I
    $
    положительно (или по крайней мере неотрицательно) определена.}

Такое условие гарантирует, что информация, содержащаяся в регрессоре $\varphi(t)$, ``достаточно богата'', и алгоритмы оценивания параметров действительно сходятся к истинному значению $\theta^\star$, а не просто обеспечивают исчезновение ошибки по выходу.

\end{document}

